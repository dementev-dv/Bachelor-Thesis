\begin{abstract}

    \begin{center}
        \large{Автоматизированная система тестирования файловой системы JFFS2} \\
    \large\textit{Дементьев Даниил Викторович} \\[1 cm]

    Современные операционные системы состоят из множества модулей, реализующих основной функционал для пользователя. Одними из основных являются модули файловых систем, отвечающие за хранение и организацию доступа к пользовательским данным. По оценкам разработчиков, на подгружаемые модули приходится большинство ошибок, приводящих к некорректной работе всей ОС. По этой причине активно разрабатываются системы тестирования модулей ядра Linux с использованием различных подходов к тестированию и анализу кода. В данной работе рассмотрена проблема тестирования файловой системы ядра Linux JFFS2 и описан процесс разработки системы автоматизированного тестирования исследуемой ФС на базе набора регрессионных тестов для различных файловых систем xfstests. В работе рассмотрены основные подходы к тестированию модулей ядра, а также использованы соответствующие различным подходам инструменты. Разработанная система покрывает 79.4\% строк исходного кода JFFS2 и 91.4\% функций. С целью дальнейшего развития разработанной системы в работе проведен анализ непокрытого кода.

    % \newpage

    % \textbf{Abstract} \\[1 cm]

    % Automative JFFS2 filesystem testing suite.
    \end{center}

\end{abstract}
\newpage
