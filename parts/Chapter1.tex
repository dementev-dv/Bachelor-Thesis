\section{Постановка задачи}
\label{sec:Chapter1} \index{Chapter1}

Целью данной работы является разработка системы автоматизированного тестирования файловой системы JFFS2.

Для достижения данной цели были поставлены следующие задачи:

\begin{itemize}
	\item Изучить особенности внутреннего устройства файловой системы JFFS2.
	\item Исследовать существующие решения тестирования файловых систем и модулей ядра Linux.
	\item Расширить тестовое покрытие кода, достигаемое существующими решениями.
	\item Разработать тесты, предназначенные для проверки особенностей JFFS2.
	\item Провести анализ достигаемого системой тестового покрытия для дальнейшего ее развития.
\end{itemize}

Актуальность данной работы обусловлена тем, что инструменты тестирования ядра Linux в лаборатории Linux Verification Center \cite{lvc} часто обнаруживают в исходном коде JFFS2 ошибки, требующие исправлений. Для проверки корректности внесенных исправлений требуется система тестирования, специализированная под особенности этой ФС и покрывающая своими тестами основной функционал.

JFFS2 в данный момент не имеет тестовых наборов, специализированных под ее архитектурные особенности ввиду небольшой популярности. Однако эта ФС до сих пор используется на множестве устройств \cite{embedded}.

\newpage
