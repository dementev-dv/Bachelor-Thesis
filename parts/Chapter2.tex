\section{Обзор существующих решений}
\label{sec:Chapter2} \index{Chapter2}

Для большинства файловых систем ядра Linux сообществом были разработаны тестовые наборы, фреймворки стресс-тестирования и инструменты фаззинг-тестирования. В данной главе будут описаны наиболее популярные и подходящие к тестированию JFFS2 решения.

\subsection{Xfstests}

\textbf{Xfstests} - набор регрессионных тестов, изначально разработанный под файловую систему xfs, но в данный момент поддерживает множество других файловых систем ядра Linux, среди которых btrfs, reiserfs, f2fs и другие. Этот набор представляет собой множество bash-скриптов, разделенных по каталогам на группы. Среди тестов имеются как нацеленные на тестирование общих для всех ФС функций, так и предназначенные для определенных систем. На данный момент xfstests является наиболее популярным и широко использующимся решением для тестирования файловых систем.

\subsection{Fsfuzz}

\textbf{Fsfuzz} - это инструмент фаззинг-тестирования ФС, представляющий собой bash-скрипт, суть работы которого заключается в генерации множества случайных образов выбранной файловой системы и дальнейшего монтирования этих образов. Подобный метод перебора не покрывает все основные функции файловой системы, однако может привести к редким сценариям обработчика ошибок, что полезно, так как подобные редкие сценарии сложно покрыть обычными тестовыми наборами.

\subsection{Syzkaller}

\textbf{Syzkaller} - наиболее популярный инструмент динамического анализа кода ядра ОС. Суть его работы заключается в конструировании случайных программ наоснове системных вызовов, запуске их на тестируемой системе и получении обратной связи в виде покрытия кода ядра. Случайные программы конструируются с использованием различных системных вызовов в различном порядке и со случайно подобранными аргументами с целью расширения покрытия кода. При достижении ошибок syzkaller генерирует программы репродьюсеры на языке Си, повторяющие сценарий достижения ошибки.

В файле конфигурации syzkaller можно определить список разрешенных к использованию системных вызовов, таким образом ограничить его покрытие, но нацелить на определенные сценарии с выбранными системными вызовами. При должной конфигурации syzkaller может быть использован для тестирования определенного модуля ядра, в том числе и ФС. 

\subsection{Тесты из пакета инструментов MTD}

Пакет \textbf{mtd-utils} содержит множество инструментов для работы с устройствами флэш-памяти, в том числе и инструмент создания образа файловой системы JFFS2. Разработчиками этого пакета инструментов были добавлены тесты для файловых систем флэш-памяти в виде четырех bash-скриптов.

Эти тесты должны хорошо покрывать функции, отвечающие особенностям файловой системы JFFS2, однако такого небольшого числа скриптов недостаточно для покрытия всех основных функций ФС.

\newpage
