\section{Заключение}
\label{sec:Chapter5} \index{Chapter5}

В результате проделанной работы была разработана система автоматизированного тестирования файловой системы JFFS2 на основе существующих решений для тестирования файловых систем. В процессе разработки системы тестирования был доработан инструмент создания образа JFFS2, путем добавления в него поддержки алгоритма сжатия данных rubin, с использованием драйвера mtdblock был адаптирован под особенности JFFS2 инструмент фаззинг-тестирования файловых систем fsfuzz. 

Набор xfstests, являющий наиболее популярным и поддерживаемым сообществом пользователей и разработчиков Linux решением для тестирования файловых систем, был адаптирован с целью тестирования JFFS2. К существующим тестовым скриптам набора были добавлены тесты пакета mtd-utils, тесты созданные из программ сгенерированных инструментом syzkaller и тесты, использующие симуляцию сбоев внутренних функций JFFS2.

Разработанная система тестирования была экспериментально проверена. В результате проведённого тестирования всеми полученными скриптами было достигнуто покрытие в 79,4\% строк исходного кода и 91,4\% функций. Это является достойным результатом и дальнейшее расширение тестового покрытия является крайне сложной задачей.

С целью дальнейшего развития разработанной системы был произведен анализ достигаемого тестового покрытия исходного кода, в результате которого были классифицированы непокрытые тестами функции ФС. По итогу проведенного анализа выявлено, что 48\% непокрытых функций являются недостижимыми при тестировании скриптами пользовательского пространства. Для тестирования еще 15\% непокрытых функций необходимо использование реальных устройств флэш-памяти определенного типа, ввиду отсутствия эмуляторов. В тестирование еще 15\% непокрытых функций нет острой необходимости, так как они фактически являются обертками над функциями общей для всех ФС библиотеки, которые могут быть протестированы другими тестовыми наборами. И оставшиеся 22\% непокрытых функций, что эквивалентно 2\% от общего числа функций файловой системы могут быть покрыты при дальнейшем развитии тестовой системы. 

Дальнейшее увеличение тестового покрытия является трудоёмкой задачей, но важно продолжать работу над улучшением качества тестирования, чтобы обеспечить более полное покрытие кода и повысить надёжность файловой системы.

\newpage
